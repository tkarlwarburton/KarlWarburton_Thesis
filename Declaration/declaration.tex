% ******************************* Thesis Declaration ***************************

\begin{declaration}

  I hereby declare that except where specific reference is made to the work of others, the contents of this thesis are original and have not been submitted in whole or in part for consideration for any other degree or qualification in this, or any other university. This thesis is my own work, and where work has been done in collaboration with others the best attempts have been made to indicate this. Any work which was performed by another individual has been expressed as such, with these figures and tables being labelled with a source. \\

  The information contained in Chapters~\ref{chap:Theory} and~\ref{chap:DUNE} are highly level summaries of the theory necessary to support this thesis, and the various facets of the DUNE experiment. These summaries have been taken from a number of articles, and are referenced as such. When discussing the LArSoft software which DUNE uses, it is necessary to point out that this is a multi-experiment project, and so many people have contributed to its development in their own small ways, including myself. \\

  Chapter~\ref{chap:Cameras} details the camera system which was installed in the 35 ton detector. The selection and basic design of the camera system was principally done by Dr. Nicola McConkey, and Matthew Thiesse, with only very minor input from the author. All work performed at Fermilab was done in conjunction with Michael Wallbank, and would have been impossible without the assistance of Dr. Linda Bagby. \\

  Chapter~\ref{chap:35tonSim} details the simulations performed on the 35 ton detector. All work presented here is my own, though it builds on large pieces of work developed by others, such as the reconstruction of both TPC and photon detector data, as well as the premise of the particle identification method which is proposed. I must thank Dr. Tingjun Yang and Dr. Tom Junk for their helpful insights in developing the methods shown here. \\

  Chapter~\ref{chap:35tonData} details all aspects of the 35 ton data which was recorded. The method of data factorisation is my own work, as is the inclusion of a Wiener filter to the data, other work is referenced to the working group presentations where they were proposed. The final section on electron diffusion is my own work, though I must thank Dr. Michelle Stancari and Dr. Dominic Brailsford for their helpful discussions. \\

  Chapter~\ref{chap:FDSims} details the far detector simulations which were performed. Section~\ref{sec:LBNESurf} details simulations performed for the LBNE surface detector, where the authors work concentrated on the implementation of the accurate detector geometry and surface profile. Simulations involving the simple geometry, and simple surface profile were performed by Dr. Martin Richardson. All work was shown in his thesis, with according references, and so all tables have been taken from there, as these had used an improved analysis to the one which we had used in partnership. The generator used for the underground location is accordingly referenced, and the generation of the sample of muons which is used was done by Dr. Matthew Robinson. The analysis presented is unique to the author, though thanks must be given to all members of the cosmogenic working group for their helpful insights. \\
  
% Author and date will be inserted automatically from thesis.tex \author \degreedate

\end{declaration}

