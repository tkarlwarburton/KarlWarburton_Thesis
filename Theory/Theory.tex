%*******************************************************************************
%******************************** Chapter XXXXXXXX *****************************
%*******************************************************************************

\chapter{Introduction and Theory} \label{chap:Theory} %Title of chapter

\newcommand{\Dmq}{\Delta m^2}
\newcommand{\eVq}{\ensuremath{\text{eV}^2}}

\graphicspath{{Theory/Figs/Raster/}{Theory/Figs/PDF/}{Theory/Figs/Vector/}}

%%% Experiments
\nomenclature[z-SK]{SK}{Super Kamiokande}

%%% Neutrino physics
\nomenclature[z-CP]{CP}{Charge-Parity}
\nomenclature[z-GUT]{GUT}{Grand Unified Theory}
\nomenclature[z-SM]{SM}{Standard Model}

%%% How LArTPCs work
\nomenclature[z-LAr]{LAr}{Liquid Argon}
\nomenclature[z-TPC]{TPC}{Time Projection Chamber}
\nomenclature[z-LArTPC]{LArTPC}{Liquid Argon Time Projection Chamber}


%%% An introduction to the standard model....
The ``Standard Model of Particle Physics'' (SM) is a set of theories which has been widely tested and has been found to accurately predict the interactions of fundamental particles. These tests have come in many forms throughout the 20$^{th}$ and 21$^{st}$ centuries, and include the observation of all of the particles which it predicts, as well as measurements of the properties of these particles. The recent discovery of the Higgs boson~\citep{HiggsAtlas, HiggsCMS} ``completed'' the SM, as this was the last particle which it predicted to be observed. However, despite it's many successes the SM does not represent the ``final'' theory of fundamental particle physics, should one exist. This is because there are many questions made by recent experimental observations which the SM is unable to address, some of these will be briefly discussed below. \\

Firstly, though the SM accurately predicts the interactions made by the electromagnetic, weak nuclear, and strong nuclear forces, it makes no mention of gravity. This is a major flaw of the SM as gravity is one of the driving forces in the formation of astronomical objects such as planets, stars, and galaxies. With the recent detection of gravitational waves~\citep{LIGO} this issue has again be brought into focus. Secondly, the rotational velocities of galaxies is measured to be far greater than the predicted value, hinting at the presence of a significant amount of matter being present which we are unable to detect. The SM makes no prediction as to what this ``dark-matter'' is comprised of. Thirdly, measurements of distant supernovae appear to show that the expansion of the universe is accelerating, not decelerating as would be expected, this implies the presence of some form of unknown energy source. Again, the SM makes no prediction as to what this unknown energy source, or ``dark-energy'' is. A further point of consternation with the SM is that it is not ``elegant,'' as it has as many as 19 free parameters, which each appear unrelated to each other. There are also unresolved questions regarding the particles which are predicted by the SM, such as, why charge is quantised, why there are exactly 3 families of quarks and leptons, and why they have the observed hierarchy of masses. The SM also does not predict the matter-antimatter asymmetry which is observed in the universe today. Finally, the neutrinos predicted by the SM are massless, however, numerous measurements of neutrino oscillation show that this is highly unlikely. This is because, it is largely accepted that in order for oscillations to occur, at least two of the neutrino flavours must have non-zero mass. A rigorous discussion of neutrino oscillations is presented in Section~\ref{sec:NeutPhys}. \\

Extensive efforts have been made to resolve many of these issues with the SM, in the form of a so-called Grand Unifying Theory (GUT). Many of these theories propose that the electroweak and strong nuclear forces belong to an overarching symmetry group. The unification of these forces is predicted to occur at extremely high energies, far beyond the reach of current experiments. As a result, many of the experimental signatures which these GUTs predict are difficult to measure. However, many GUTs predict that the proton, a stable particle in the SM, should decay with a lifetime of the order of 10$^{34}$ years. Some of the more common mechanisms by which GUTs are predicted are briefly discussed in Section~\ref{sec:Theory_GUT}, with reference to the proton lifetimes which they predict. A discussion of the backgrounds to proton decay searches is presented in Section~\ref{sec:BkNDK}. \\

The Deep Underground Neutrino Experiment (DUNE) is a next generation experiment situated at the Sanford Underground Research Facility (SURF), which aims to measure many of the properties of neutrinos, as well as to search for nucleon decays. The experimental setup, physics capabilities, and prototyping schedule for DUNE are outlined in Chapter~\ref{chap:DUNE}. A camera system which was installed in the DUNE 35 ton prototype detector is described in Chapter~\ref{chap:Cameras}. Chapter~\ref{chap:35tonSim} describes simulations which were made in preparation for data taking of the 35 ton prototype, and conclude with a description of how Particle Identification (PID) could be performed in the 35 ton data. An overview of the data gathered by the 35 ton prototype is shown in Chapter~\ref{chap:35tonData}, and a novel method of interaction time determination using the effects of diffusion is presented. Following this, Chapter~\ref{chap:FDSims} concerns simulations of the cosmogenic backgrounds seen in large Liquid Argon Time Projection Chambers (LArTPCs) detectors at SURF. These simulations are first presented with respect to a surface detector measuring neutrino oscillations, and then to a detector at depth searching for nucleon decay events. Finally, Chapter~\ref{chap:Conc} contains some final remarks and observations. \\

%********************************** %First Section  **************************************
\section{Neutrino physics} ~\label{sec:NeutPhys}  %Section - X.1 
As alluded to earlier, the study of neutrinos offers a chance to probe the limitations of the SM. This is because the neutrinos predicted by the SM are massless and do not oscillate. However, numerous measurements have shown that neutrino oscillation occurs, and that at least two of the neutrino flavours are massive. Notably, the 2015 Nobel prize in physics was given to T. Kajita and A. McDonald for ``the discovery of neutrino oscillations, which shows that neutrinos have mass,'' for their work on Super-Kamiokande (SK)~\citep{PhysRevLett.81.1562} and SNO~\citep{PhysRevLett.89.011301} respectively. This means that through studying neutrino oscillations, it is possible to begin to get a handle on physics beyond the SM. The history of the discovery of neutrino oscillations, which culminated in this Nobel prize, is briefly outlined in Section~\ref{Neut_Hist}. Following this, the formalism by which neutrino oscillation occurs is presented in Section~\ref{Neut_Oscil}. Finally, the current state of neutrino physics, including the current best fit values for the various mixing parameters, is summarised in Section~\ref{sec:Theory_Exp}. \\

%%% The history
\subsection{The history of neutrino oscillation} \label{Neut_Hist}
Neutrinos were first proposed to explain the continuous energy spectrum of the electrons produced in $\beta$ decay, as due to kinematic constraints, it could not be explained by a two body decay. To this end, Pauli proposed the idea of a neutral particle, with mass less than that of the electron, which would not be observed in the reaction~\citep{Pauli}. Pauli called this particle a ``neutron.'' Upon the discovery that the ``neutron'' was in fact of a similar mass to the proton, and that the nucleus was a bound state of protons and neutrons, Fermi proposed a more complete theory of $\beta$ decay in 1934. In this theory, Fermi proposed that the light, neutral particle that was initially proposed by Pauli did exist, and was emitted from the nucleus in the reaction. Fermi called this particle a ``neutrino,'' meaning ``little neutral one'' in Italian. He also proposed that its mass could be measured by looking at the end point of the $\beta$ spectrum~\citep{Fermi:1934hr}. The first experiments designed to measure the neutrino mass in this way set an upper mass limit of 500 eV~\citep{NeutMassLim1, NeutMassLim2}, which was improved to 250 eV in the 1950's\citep{NeutMassLim3}. After becoming evident that the neutrino mass was so much less than that of the electron, the idea that neutrinos were massless gained traction. \\

The existence of neutrinos was confirmed in 1956~\citep{Cowan:1992xc}, and in 1962 conclusive proof emerged that the electron and muon neutrinos were distinct particles~\citep{PhysRevLett.9.36}. The experiment which found this, did so by observing that it was far more likely for muons to be produced in the decay of pions, as opposed to electrons. This meant that the pion coupled more strongly to the muon, and so there had to exist two distinct particles, each with a different coupling to the pion. However, soon after this in 1968, an experiment by Ray Davis at the Homestake Mine gave rise to the ``Solar neutrino problem''~\citep{RayDavis1968}. The Homestake experiment used a chlorine detector to look for the electron neutrinos produced by the sun, and measured a flux which was roughly $\frac{1}{3}$ of the predicted flux from solar models. The experiment ran for over 20 years, with the measured $v_{e}$ flux being unchanged, at roughly $\frac{1}{3}$ of the predicted solar flux~\citep{RayDavis1988}. \\

The long standing observation that the solar $v_{e}$ flux was significantly lower than predicted, meant that either, there was some mechanism by which the electron neutrinos were evading detection, or that the solar model was incorrect. It was plausible that the solar model was incorrect, however the scale of the difference in the observed, and predicted, $v_{e}$ flux proved difficult to resolve. As a result, the idea of neutrino oscillation grew momentum, drawing on a prediction made by Pontecorvo as far back as 1957~\citep{Pontecorvo1957}. The SK experiment measured high energy solar neutrinos, and in 1989 measured an energy dependant deficit in the solar $v_{e}$ flux~\citep{PhysRevLett.63.16}. When studying the atmospheric neutrino flux, SK found an angular dependent deficit in the expected muon neutrino flux, though the electron neutrino flux was consistent with predictions. It was found that this deficit was consistent with oscillations of $\nu_{\mu} \leftrightarrow \nu_{\tau}$~\citep{PhysRevLett.81.1562}. The existence of the third flavour of leptons has been shown in the 1970s~\citep{PhysRevLett.35.1489}, though the $\nu_{\tau}$ neutrino was itself not directly measured until 2000~\citep{Kodama2001218}. \\

Conclusive proof for neutrino oscillations came in 2001, when the SNO experiment measured both the Neutral Current (NC) and Charged Current (CC) interactions of solar neutrinos. The SNO experiment measured a charge current interaction rate which was consistent with the experiments which had gone before it~\citep{Ahmad:2001an}, i.e. a deficit in the predicted solar flux. However, it found that the neutral current interaction rate, which is sensitive to all flavours of neutrinos, matched the predicted solar flux~\citep{PhysRevLett.89.011301}. This demonstrated that a significant part of the $v_{e}$ flux from the sun, had oscillated into $v_{\mu}$ and $v_{\tau}$ as they travelled to Earth. This flux of oscillated $v_{\mu}$ and $v_{\tau}$ could not interact via CC interactions, due to the high mass of the associated leptons relative to the neutrino energy, however, they are able to interact via NC interactions. \\

These highlighted results, as well as many other accompanying results, form the basis of our current understanding of neutrino oscillations. This is explained in Section~\ref{Neut_Oscil}. \\

%%% The formalism...
\subsection{The theory of neutrino oscillations} \label{Neut_Oscil}
Neutrino oscillations are described by the PMNS matrix, which is named after work initially done by Pontecorvo~\citep{Pontecorvo1957}, and then later extended by Maki, Nakagawa and Sakata~\citep{PMNS}. The PMNS matrix describes neutrino mixing in the context of the three known flavour states $\nu_e$, $\nu_\mu$ and $\nu_\tau$ being related to three neutrino mass states $\nu_1$, $\nu_2$ and $\nu_3$. This PMNS then has the form shown in Equation~\ref{eq:PMNS_Short}.
\begin{equation}
  \label{eq:PMNS_Short}
  \begin{pmatrix} \nu_e \\ \nu_\mu \\ \nu_\tau \end{pmatrix} = U_{PMNS} \begin{pmatrix} \nu_1 \\ \nu_2 \\ \nu_3 \end{pmatrix}
\end{equation}
The matrix labelled $U_{PMNS}$ in Equation~\ref{eq:PMNS_Short}, is then expressed as is shown in Equations~\ref{eq:PMNS_El},~\ref{eq:PMNS_Who} and~\ref{eq:PMNS_Exp}. In these equations the abbreviations $s_{\alpha\beta}$ and $c_{\alpha\beta}$ are used for $\sin_{\alpha\beta}$ and $\cos_{\alpha\beta}$ respectively. A CP-violating phase $\delta$ is also introduced. \\
\begin{align}
  U_{PMNS} &= \begin{pmatrix} U_{e1} & U_{e2} & U_{e3} \\ U_{\mu1} & U_{\mu2} & U_{\mu3} \\ U_{\tau1} & U_{\tau2} & U_{\tau3} \end{pmatrix} \label{eq:PMNS_El} \\
  &= \begin{pmatrix} c_{12}c_{13}                                  & s_{12}c_{13}                                  & s_{13}e^{-i\delta} \\
                     -s_{12}c_{23} - c_{12}s_{23}s_{13}e^{i\delta} & c_{12}c_{23} - s_{12}s_{23}s_{13}e^{i\delta}  & s_{23}c_{13}      \\
                     s_{12}s_{23} - c_{12}s_{23}s_{13}e^{i\delta}  & -c_{12}s_{23} - s_{12}c_{23}s_{13}e^{i\delta} & c_{23}c_{13}      \end{pmatrix} \label{eq:PMNS_Who}  \\
  &= \begin{pmatrix} 1 & 0 & 0                       \\ 0 & c_{23} & s_{23}  \\ 0 & -s_{23} & c_{23}            \end{pmatrix}
     \begin{pmatrix} c_{13} & 0 & s_{13}e^{-i\delta} \\ 0 & 1 & 0            \\ -s_{13}e^{i\delta} & 0 & c_{13} \end{pmatrix}
     \begin{pmatrix} c_{12} & s_{12} & 0             \\ -s_{12} & c_{12} & 0 \\ 0 & 0 & 1                       \end{pmatrix} \label{eq:PMNS_Exp}
\end{align}

Equation~\ref{eq:PMNS_El} shows how each element in the $U_{PMNS}$ relates the flavour states to the mass states, whilst Equation~\ref{eq:PMNS_Who} shows the full mixing formalism for Dirac neutrinos. Finally, Equation~\ref{eq:PMNS_Exp} separates the full formalism into three 3$\times$3 matrices which each contain one of the three mixing angles. Should neutrinos be Majorana particles, then the $U_{PMNS}$ matrices should be multiplied by diag$\left(e^{i\alpha_1/2}, e^{i\alpha_2/2}, 1\right)$. This results in the neutrino mixing matrix being constrained by six independent parameters:
\begin{itemize}
 \item Three mixing angles ($\theta_{13}$, $\theta_{12}$, $\theta_{23}$).
 \item The CP-violating phase ($\delta$).
 \item Two Majorana phases ($\alpha_1$, $\alpha_2$).
\end{itemize}
However, the two Majorana phases do not affect neutrino oscillations, and so will not be covered extensively here. The question of whether neutrinos are Majorana or Dirac particles does raise important questions for neutrino physics though. This is because should neutrinos be Majorana particles, their masses could be generated via a Majorana mass term. There are next generation experiments such as SNO+~\citep{SNO+} and SuperNEMO~\citep{SuperNEMO}, which will search for neutrinoless double beta decay as a means to test whether neutrinos are Majorana particles. \\

When neutrinos are produced and detected, we observe neutrinos of distinct flavour states, and not the distinct mass states. Therefore, a discussion of how mixing occurs will be presented in terms of an initial neutrino composed of a distinct flavour state, and multiple mass states. In this case a neutrino $\nu_\alpha$ will be produced, of flavour $\alpha$, which is a linear superposition of the three mass eigenstates, labelled $\nu_j$ (Equation~\ref{eq:Oscill_Eigen}). 
\begin{equation}
  \label{eq:Oscill_Eigen}
  \ket{\nu_\alpha} = \sum_{j}U^{\ast}_{\alpha j}\ket{\nu_j}
\end{equation}
As this neutrino propagates, the mass eigenstates will evolve according to the time-dependant Schr\"{o}dinger equation, such that after time $t$, each mass eigenstate will have the form shown in Equation~\ref{eq:Oscill_MassProp1}.
\begin{equation}
  \label{eq:Oscill_MassProp1}
  \ket{\nu_j(t)} = e^{-i(E_{j}\cdot t - \vec{p_{j}}\cdot\vec{x_{j}})}\ket{\nu_j(0)}
\end{equation}
where assuming that the neutrino is ultra-relativistic:
\begin{align}
  t &\approx L \label{eq:Oscill_t2L} \\
  E &= \sqrt{p^2+m^2} = p \times \sqrt{1+\frac{m^2}{p^2}} \approx p + \frac{m^2}{2p} \approx p + \frac{m^2}{2E} \label{eq:Oscill_ESub} \\
  E_{j}\cdot t - \vec{p_{j}}\cdot\vec{x_{j}} &\approx \vec{p_{j}}L + \frac{m_{j}^2}{2E}L - \vec{p_{j}}L = \frac{m_{j}^2}{2E}L \label{eq:Oscill_FinSub}  
\end{align}
where a Taylor series expansion about $\sqrt{1+x^2}$ has been used in Equation~\ref{eq:Oscill_ESub}. Substituting Equation~\ref{eq:Oscill_FinSub} into Equation~\ref{eq:Oscill_MassProp1} gives Equation~\ref{eq:Oscill_MassProp2}.
\begin{equation}
  \label{eq:Oscill_MassProp2}
  \ket{\nu_j(t)} = e^{-im_{j}^{2}L/2E}\ket{\nu_j(0)}
\end{equation}
This then gives the time evolution of the original neutrino flavour state as Equation~\ref{eq:Oscill_TimeEigen}.
\begin{equation}
  \label{eq:Oscill_TimeEigen}
  \ket{\nu_\alpha(t)} = \sum_{j}U^{\ast}_{\alpha j}e^{-im_{j}^{2}L/2E}\ket{\nu_j(0)}
\end{equation}
From this, it can be seen that the mass states propagate with different phases, and so should the neutrino be detected at a later time it would exist as a superposition of different flavour states. This results in there being a non-zero possibility that the flavour of the neutrino which is detected, $\beta$, is not the same as the flavour with which the neutrino was produced with, $\alpha$. The amplitude with which this occurs is given by Equation~\ref{eq:Oscill_Amp}.
\begin{align}
    A(\nu_{\alpha}\rightarrow\nu_{\beta}) &= \bra{\nu_{\beta}}\ket{\nu_\alpha(t)} \nonumber \\
    &= \sum_{k}\sum_{j}\bra{\nu_{j}}U_{\beta j}U^{\ast}_{\alpha k}e^{-im_{k}^{2}L/2E}\ket{\nu_{k}} \nonumber \\
    &= \sum_{k}U^{\ast}_{\alpha k}U_{\beta k}e^{-im_{k}^{2}L/2E}   \label{eq:Oscill_Amp}
\end{align}
Equation~\ref{eq:Oscill_Amp}, can then be used to get the probability for the original neutrino $\nu_{\alpha}$ to oscillate to a different flavour $\nu_{\beta}$ is then given by Equation~\ref{eq:Oscill_Prob2}.
\begin{align}
  P(\nu_{\alpha}\rightarrow\nu_{\beta}) &= |\bra{\nu_{\beta}}\ket{\nu_\alpha(t)}|^2 \nonumber \\
  &= \left|\sum_{k}U^{\ast}_{\alpha k}U_{\beta k}e^{-im_{k}^{2}L/2E}\right|^2 \label{eq:Oscill_Prob1}\\
  &= \sum_{k}U^{\ast}_{\alpha k}U_{\beta k}e^{-im_{k}^{2}L/2E} \sum_{j}U^{\ast}_{\alpha j}U_{\beta j}e^{-im_{j}^{2}L/2E} \nonumber\\
  &= \sum_{k}\sum_{j}U^{\ast}_{\alpha k}U_{\beta k}U^{\ast}_{\alpha j}U_{\beta j}exp\left(-i\frac{(m^{2}_{k}-m^{2}_{j})L}{2E}\right) \label{eq:Oscill_Prob2}
\end{align}
The $(m^{2}_{k}-m^{2}_{j})$ term in Equation~\ref{eq:Oscill_Prob2} is often written as $\Delta m^{2}_{kj}$, and will be written as such for the remainder of the discussion of neutrino mixing. \\

In the interests of simplicity, an explicit calculation of the neutrino oscillation probability will be given assuming that there are only two neutrino flavour and mass states. The reason for this is that then there is only one mixing angle ($\theta$), and no complex phase. We are also free to choose the simplest mixing matrix, such that the mixing matrix becomes Equation~\ref{eq:PMNS_Simp}.
\begin{equation}
  \label{eq:PMNS_Simp}
  \begin{pmatrix} \nu_{\alpha} \\ \nu_{\beta} \end{pmatrix} = \begin{pmatrix} \cos\theta & \sin\theta \\ -\sin\theta & \cos\theta \end{pmatrix} \begin{pmatrix} \nu_1 \\ \nu_2 \end{pmatrix}
\end{equation}
The probability of a neutrino oscillating from initial flavour $\nu_{\alpha}$ to flavour $\nu_{\beta}$ is then given by Equation~\ref{eq:Oscil_TwoNeut}, which starts from Equation~\ref{eq:Oscill_Prob1}.
\begin{align}
  P(\nu_{\alpha}\rightarrow\nu_{\beta}) &= \left|(U_{\alpha 1}U_{\beta 1}e^{-im_{1}^{2}L/2E}) + (U_{\alpha 2}U_{\beta 2}e^{-im_{2}^{2}L/2E})\right|^2 \nonumber \\
  &= \left|(\cos\theta)(-\sin\theta)e^{-im_{1}^{2}L/2E} + (\sin\theta)(\cos\theta)e^{-im_{2}^{2}L/2E}\right|^2 \nonumber \\
  &= 2\cos^{2}\theta\sin^{2}\theta - \cos^{2}\theta\sin^{2}\theta\left[ e^{-i\frac{(m^{2}_{1}-m^{2}_{2})L}{2E}} + e^{-i\frac{(m^{2}_{2}-m^{2}_{1})L}{2E}} \right] \nonumber\\ 
  &\text{using $\cos\left(\phi_1 - \phi_2\right) = \left(e^{i(\phi_1-\phi_2)} + e^{-i(\phi_1-\phi_2)}\right)/2$} \nonumber \\
  &= 2\cos^{2}\theta\sin^{2}\theta - \cos^{2}\theta\sin^{2}\theta\left[ 2\cos\left(\frac{(m^{2}_{1}-m^{2}_{2})L}{2E} \right) \right] \nonumber \\
  &= 2\cos^{2}\theta\sin^{2}\theta \left[1 - cos\left(\frac{\Delta m^{2}_{12}L}{2E}\right) \right] \nonumber \\
  &\text{using $\cos\theta\sin\theta = \frac{1}{2}\sin(2\theta)$ and $2\sin^{2}(\theta) = 1 - \cos(2\theta)$} \nonumber \\
  &= \sin^{2}2\theta \sin^{2}\left(\frac{\Delta m^{2}_{12}L}{4E}\right) \label{eq:Oscil_TwoNeut} \\
  &\text{Measuring $L$ in km, $E$ in GeV, and considering factors of $\hbar$ and $c$} \nonumber \\
  &= \sin^{2}2\theta \sin^{2}\left(1.27\Delta m^{2}_{12}\frac{L}{E}\right) \label{eq:Oscil_Normalised}
\end{align}
The presence of the $\Delta m^{2}_{kj}$ terms in Equations~\ref{eq:Oscill_Prob2} and~\ref{eq:Oscil_TwoNeut} is the reason why neutrino oscillation implies that at least two neutrinos are massive. This is because if this term is 0, then the probability of oscillation is 0, and so oscillations would not occur. \\

It can often be assumed that there are only two flavours of neutrinos, as recent experimental data shows that the mass splittings of the three neutrino flavours are separated by many orders of magnitude. The current best fit experimental values are shown in Section~\ref{sec:Theory_Exp}, though it is sufficient to say $\Delta m^{2}_{23} \approx \Delta m^{2}_{13} \gg \Delta m^{2}_{12}$. For example, in the case of atmospheric neutrinos, oscillations are largely due to $\nu_{\mu}\rightarrow\nu_{\tau}$. When explaining the observed deficit of upward going $\nu_{\mu}$ the $L/E$ of oscillations must be considered, as the other terms in Equation~\ref{eq:Oscil_Normalised} are all constants. When considering an initial $\nu_{\mu}$ of energy 1 GeV, the oscillation probability will be small for down-going neutrinos (L = 10 km), however for upwards-going muons (L = 10$^4$ km) the oscillation probability is much larger, at around 50\%. This is what is seen in experimental data. \\

When measuring neutrino oscillations using neutrinos produced by accelerators, one also has to consider the most optimal $L/E$. These experiments measure either the disappearance, or appearance of neutrino flavour states. MINOS(+)\citep{Tzanankos:2011zz} is an example of a disappearance experiment, it measured the ratio of $\nu_{\mu}$ neutrinos at magnetised near and far detectors, both of which were made of thick iron plates interspersed with scintillator bars. However, appearance experiments such as DUNE primarily measure the appearance of $\nu_{e}$ from an initially pure $\nu_{\mu}$ beam. \\

There are of course three flavours of neutrinos, not two as assumed above. The addition of an extra neutrino mass and flavour state introduces added complexities to the oscillation probabilities, due to the extra mixing angle and the complex phase which are introduced. The oscillation probability for the three flavour neutrino case is shown in Equation~\ref{eq:Oscill_3Flav}.
\begin{align}
  \begin{split}
    \label{eq:Oscill_3Flav}
    P(\nu_{\alpha}\rightarrow\nu_{\beta}) = \delta_{\alpha\beta}
    & - 4 \sum_{i<j}\Re(U^{\ast}_{\alpha i}U_{\alpha j}U_{\beta i}U_{\beta j}) \sin^2\left(\frac{\Delta m^{2}_{ij}}{4E}L\right) \\
    & - 2 \sum_{i<j}\Im(U^{\ast}_{\alpha i}U_{\alpha j}U_{\beta i}U_{\beta j}) \sin\left(2\frac{\Delta m^{2}_{ij}}{4E}L\right)
  \end{split}
\end{align}
Equation~\ref{eq:Oscill_3Flav} considers oscillations in a vacuum, however DUNE will measure $\nu_{\mu}\rightarrow\nu_{e}$ oscillations after the neutrinos have passed through a large amount of matter. Measuring neutrinos which have passed through matter adds further complications to the observed oscillation probability. The full oscillation probability for $\nu_{\mu}\rightarrow\nu_{e}$ oscillations, after they have passed through matter of constant density, is shown in Equation~\ref{eq:Oscill_3FlavFull}~\citep{Nunokawa:2007qh}.
\begin{align}
  P(\nu_{\mu}\rightarrow\nu_{e}) \simeq
  & \sin^{2}\theta_{23} \sin^{2}2\theta_{13} \frac{sin^{2}(\Delta_{31}-aL)}{(\Delta_{31}-aL)^2}\Delta^{2}_{31} \label{eq:Oscill_3FlavFull} \\
  &+ \sin2\theta_{23} \sin2\theta_{13} \sin2\theta_{12} \frac{sin(\Delta_{31}-aL)}{(\Delta_{31}-aL)}\Delta_{31} \frac{sin(aL)}{(aL)} \Delta_{12} \cos(\Delta_{31}+\delta_{CP}) \nonumber \\
  &+ \cos^{2}\theta_{23} \sin^{2}2\theta_{12} \frac{sin^{2}(aL)}{(aL)^2}\Delta^{2}_{12} \nonumber
\end{align}
where $\Delta_{ij} = \Delta m^{2}_{ij}L/4E$, $\delta_{CP}$ is the CP-violating phase, and $a = G_{F}N_{e}/\sqrt{2}$ with $G_{F}$ being the Fermi constant and $N_{e}$ being the number density of electrons in the Earth. From Equation~\ref{eq:Oscill_3FlavFull} it can be seen that the probability of neutrino oscillations is governed by the following parameters:
\begin{itemize}
 \item Three mixing angles ($\theta_{13}$, $\theta_{12}$, $\theta_{23}$).
 \item Three mass squared differences ($\Delta m^{2}_{12}$, $\Delta m^{2}_{13}$, $\Delta m^{2}_{23}$).
 \item The CP-violating phase ($\delta$).
 \item The distance the neutrino travels ($L$).
 \item The energy of the neutrino ($E$).
 \item The magnitude of the matter effects ($a$).
\end{itemize}
Of these parameters, the mixing angles, mass squared differences, and $\delta_{CP}$ are fixed and can only be measured. The distance travelled by the neutrino, and the neutrino energy vary, though may be chosen by the experiment, in the case of accelerator or reactor based experiments. The magnitude of the matter effects cannot be controlled by the experiment, and arises naturally when neutrinos travel through matter. The CP-violating phase ($\delta_{CP}$), and the matter effect ($a$), have been introduced in Equation~\ref{eq:Oscill_3FlavFull} and will be briefly discussed below. \\

%%%% DUNE CDR Volume 2 has a good overview of CP stuff....
CP violation occurs when a particle and its anti-particle behave differently. This has been observed in the quark sector, where the CP violating phase $\delta^{CKM}_{CP}$ has been measured to be approximately 70$^{\circ}$~\citep{PDGReview}. Despite this large CP-violating phase (90$^{\circ}$ representing a maximal CP violating phase), CP-violation in the quark sector is relatively small due to the small mixing present in the CKM matrix~\citep{PhysRevLett.10.531, Kobayashi:1973fv}. The CKM matrix describes quark mixing, in the same way that the PMNS matrix describes neutrino mixing. This lack of CP-violation makes it difficult to explain the observed matter-antimatter asymmetry in the Universe, and so it is widely hoped that measuring a large leptogenic CP-violation may help to explain this. A large amount of CP-violation is possible in the neutrino sector due to the large mixing angles. \\ 

It can be seen from Equation~\ref{eq:Oscill_3FlavFull} that in order for $\delta_{CP}$ to be measured, all three mixing angles ($\theta_{ij}$) must be nonzero. This has now been confirmed by experiments, as will be shown in Section~\ref{sec:Theory_Exp}. CP-violation in neutrinos can be observed by measuring the differences in the oscillations of $\nu_{\mu}\rightarrow\nu_{e}$ and $\overline{\nu_{\mu}}\rightarrow\overline{\nu_{e}}$. The asymmetry between neutrino and antineutrino oscillations is given by Equation~\ref{eq:Oscill_ACP}.
\begin{equation}
  \label{eq:Oscill_ACP}
  A_{CP} = \frac{ P(\nu_{\mu}\rightarrow\nu_{e}) - P(\overline{\nu_{\mu}}\rightarrow\overline{\nu_{e}}) } { P(\nu_{\mu}\rightarrow\nu_{e}) + P(\overline{\nu_{\mu}}\rightarrow\overline{\nu_{e}}) }
\end{equation}
However, the observed asymmetry between $\nu_{\mu}\rightarrow\nu_{e}$ and $\overline{\nu_{\mu}}\rightarrow\overline{\nu_{e}}$ oscillations will also be affected by matter effects, and so they must also be accurately understood before the value of $\delta_{CP}$ can be calculated. \\

The Mikheyev-Smirnov-Wolfenstein (MSW) effect~\citep{PhysRevD.17.2369, Mikheev:1986gs}, or matter effect, causes the effective mass of neutrinos to increase due to coherent scattering. The increase in the effective mass of each neutrino is different though, and means that the effective mass splittings in matter become different from those in a vacuum. This leads to the possibility of measuring the absolute mass differences between different mass states, note that the oscillation probabilities presented earlier were only sensitive to the squared mass differences. The increase in the effective mass of the $\nu_{e}$ is largest, as it is able to undergo charged current interactions with electrons in matter, note that this is not the case for $\overline{\nu_{e}}$ as there are few positrons in matter. This leads to the MSW effect being able to mimic the effects of CP violation. \\

The MSW effect is observed in solar neutrinos, where the $\nu_{e}$ which are produced in the core oscillate to $\nu_{\mu}$ as they travel through dense regions of matter in the sun, causing the oscillation probability to increase. By comparing measurements made by experiments looking at solar and reactor neutrinos, it can be seen that the $\nu_{2}$ mass state is heavier than the $\nu_{1}$ mass state. This is because if the $\nu_{1}$ mass state was heavier, the oscillation probability would decrease, not increase. \\

However, it has not yet been determined whether the $\nu_{3}$ mass state is lighter or heavier than the $\nu_{1,2}$ mass states. This results in there being two potential mass hierarchies, a normal mass hierarchy where the mass states are ordered such that $\nu_{{1}} < \nu_{{2}} < \nu_{{3}}$, or an inverted mass hierarchy where the mass states are ordered such that $\nu_{{3}} < \nu_{{1}} < \nu_{{2}}$. The normal and inverted hierarchies are sometimes also referred to as normal and inverted orderings. Figure~\ref{fig:Neut_MH} is a schematic representation of the mass hierarchies, showing the fractional component of flavour states in each mass state. \\
\begin{figure}
  \centering
  \includegraphics[width=0.5\textwidth]{MassHierarchy}
  \caption[A schematic representation of the two mass hierarchies, showing the fractional components of flavour states in each mass state]
          {A schematic representation of the two mass hierarchies, showing the fractional components of flavour states in each mass state. Left: the normal hierarchy, where the masses are ordered $\nu_{{1}} < \nu_{{2}} < \nu_{{3}}$. Right: the inverted mass hierarchy, where the masses are ordered $\nu_{{3}} < \nu_{{1}} < \nu_{{2}}$. The fractional components of $\nu_{e}$, $\nu_{\mu}$ and $\nu_{\tau}$ are shown in red, green and blue respectively. Figure taken from~\citep{Hewett:2012ns}.}
  \label{fig:Neut_MH}
\end{figure}

%%% Current best limits and overview of exisiting experiments.
\subsection{Current experimental limits, and unanswered questions} \label{sec:Theory_Exp}
There is a large amount of experimental data supporting the oscillation paradigm which has been outlined in Section~\ref{Neut_Oscil}. This is summarised in Table~\ref{tab:NeutProp}~\citep{NuFit2016}, which combines measurements made by many neutrino experiments to produce global fits for the neutrino mixing parameters. \\ 

\begin{table}\centering
  \resizebox{\columnwidth}{!}{%
    \begin{tabular}{l|cc|cc|c}
      \hline\hline
      & \multicolumn{2}{c|}{Normal Ordering (best fit)}
      & \multicolumn{2}{c|}{Inverted Ordering ($\Delta\chi^2=0.83$)}
      & Any Ordering
      \\
      \hline
      & bfp $\pm 1\sigma$ & $3\sigma$ range
      & bfp $\pm 1\sigma$ & $3\sigma$ range
      & $3\sigma$ range
      \\
      \hline
      \rule{0pt}{4mm}\ignorespaces
      $\sin^2\theta_{12}$
      & $0.306_{-0.012}^{+0.012}$ & $0.271 \to 0.345$
      & $0.306_{-0.012}^{+0.012}$ & $0.271 \to 0.345$
      & $0.271 \to 0.345$
      \\[1mm]
      $\theta_{12}/^\circ$
      & $33.56_{-0.75}^{+0.77}$ & $31.38 \to 35.99$
      & $33.56_{-0.75}^{+0.77}$ & $31.38 \to 35.99$
      & $31.38 \to 35.99$
      \\[3mm]
      $\sin^2\theta_{23}$
      & $0.441_{-0.021}^{+0.027}$ & $0.385 \to 0.635$
      & $0.587_{-0.024}^{+0.020}$ & $0.393 \to 0.640$
      & $0.385 \to 0.638$
      \\[1mm]
      $\theta_{23}/^\circ$
      & $41.6_{-1.2}^{+1.5}$ & $38.4 \to 52.8$
      & $50.0_{-1.4}^{+1.1}$ & $38.8 \to 53.1$
      & $38.4 \to 53.0$
      \\[3mm]
      $\sin^2\theta_{13}$
      & $0.02166_{-0.00075}^{+0.00075}$ & $0.01934 \to 0.02392$
      & $0.02179_{-0.00076}^{+0.00076}$ & $0.01953 \to 0.02408$
      & $0.01934 \to 0.02397$
      \\[1mm]
      $\theta_{13}/^\circ$
      & $8.46_{-0.15}^{+0.15}$ & $7.99 \to 8.90$
      & $8.49_{-0.15}^{+0.15}$ & $8.03 \to 8.93$
      & $7.99 \to 8.91$
      \\[3mm]
      $\delta_\text{CP}/^\circ$
      & $261_{-59}^{+51}$ & $\hphantom{00}0 \to 360$
      & $277_{-46}^{+40}$ & $145 \to 391$
      & $\hphantom{00}0 \to 360$
      \\[3mm]
      $\dfrac{\Dmq_{21}}{10^{-5}~\eVq}$
      & $7.50_{-0.17}^{+0.19}$ & $7.03 \to 8.09$
      & $7.50_{-0.17}^{+0.19}$ & $7.03 \to 8.09$
      & $7.03 \to 8.09$
      \\[3mm]      
      $\dfrac{\Dmq_{3\ell}}{10^{-3}~\eVq}$
      & $+2.524_{-0.040}^{+0.039}$ & $+2.407 \to +2.643$
      & $-2.514_{-0.041}^{+0.038}$ & $-2.635 \to -2.399$
      & $\begin{bmatrix}
        +2.407 \to +2.643\\[-2pt]
        -2.629 \to -2.405
      \end{bmatrix}$
      \\[3mm]
      \hline\hline
    \end{tabular}%
  }
  \caption[Three-flavor oscillation parameters from a fit to global data after the NOW~2016 and ICHEP-2016 conference]
          {Three-flavor oscillation parameters from a fit to global data after the NOW~2016 and ICHEP-2016 conference. The numbers in the 1st (2nd) column are obtained assuming normal (inverted) ordering, whereas in the 3rd column no ordering is assumed. Note that $\Dmq_{3\ell} \equiv \Dmq_{31} > 0$ for normal ordering, and $\Dmq_{3\ell} \equiv \Dmq_{32} < 0$ for inverted ordering. Table is taken in full from~\citep{NuFit2016}.}
  \label{tab:NeutProp}
\end{table}

As can be seen from Table~\ref{tab:NeutProp}, the three mixing angles, and the mass squared differences are known to quite high precision. This could lead one to believe that the physics of neutrino oscillation is completely understood, however there are still many questions which remain unanswered. Some of these have already been discussed earlier, and in no particular order are;
\begin{enumerate}
\item \textbf{What is the quadrant of $\theta_{23}$?} As can be seen from Table~\ref{tab:NeutProp}, $\theta_{23}$ is very close to 45$^{\circ}$. However, whether it is more or less than 45$^{\circ}$ remains to be seen.
\item \textbf{What is the value of $\delta_{CP}$?} The value of $\delta_{CP}$ is largely unconstrained to 3$\sigma$, and as discussed in Section~\ref{Neut_Oscil}, the observation of CP violation in neutrinos may help to explain the matter-antimatter asymmetry in the universe.  
\item \textbf{What is the neutrino mass hierarchy?} As discussed in Section~\ref{Neut_Oscil}, though it is known that the $\nu_{1}$ mass state is lighter than the $\nu_{2}$ mass state, it is not known whether these are lighter or heavier than the $\nu_{3}$ mass state.
\item \textbf{What is the absolute neutrino mass scale?} The absolute mass of the neutrino is unknown, though cosmological constraints, from Planck CMB data, place an upper limit of $\sum m_{\nu} < 0.23$ eV at the 95\% confidence limit~\citep{Planck}. Neutrino experiments such as KATRIN hope to reach sensitivities of this magnitude by looking for a cut-off in the energy spectrum of $\beta$ decays~\citep{KATRIN}.
\item \textbf{Are neutrinos Dirac or Majorana particles?} This concerns how neutrinos acquire mass, as if they are Dirac particles then they would acquire mass through interactions with the Higgs field, as the other particles in the SM do. However, if they were Majorana particles, then they would acquire at least some of their mass through self-coupling. This would violate lepton number, and mean that neutrinos are there own anti-particle. As discussed in Section~\ref{Neut_Oscil}, experiments are searching for this by looking for neutrinoless double $\beta$ decay.
\item \textbf{Are there sterile neutrinos?} Experimental data from LEP tightly constrains the number of active neutrinos to those predicted by the SM~\citep{LEP}. However, in recent years there have been anomalous results which may hint at so-called sterile neutrinos~\citep{LSND1, LSND2, MiniBooNE}. These sterile neutrinos would interact rarely, or not at all, but would still be involved in oscillations. Sterile neutrinos are included in the neutrino mixing formalism by extending the PMNS matrix from a 3 $\times$ 3 matrix, to a (3+N) $\times$ (3+N) matrix, where N is the number of sterile neutrinos. Many short baseline experiments are searching for signatures which would suggest the presence of sterile neutrinos.  
\end{enumerate}
Next generation neutrino experiments, such as DUNE, hope to address many of these unanswered questions. \\

%********************************** % Second Section  *************************************
\section{Grand Unifying Theories and nucleon decay}  \label{sec:Theory_GUT} %Section - X.2
The subject of Grand Unifying Theories (GUTs) is a complicated one, and requires a level of mathematics which will not be covered here. However, as Chapter~\ref{chap:FDSims} concerns establishing background limits to nucleon decays it is necessary to illustrate the basic premise of GUTs, and the predictions on nucleon decay which they make. The following information is a summary of~\citep{Senjanovic:2009kr}. \\

The basic premise of GUTs is that they attempt to unite the strong, weak and electromagnetic forces, this is achieved by referring to very large energy scales of around 10$^16$ GeV. One of the first GUTs was that of Georgi and Glashow in 1974, which predicted that the three forces arose from a single interaction based on the SU(5) gauge group~\citep{PhysRevLett.32.438}. One of the things which their theory predicted was that the proton would not be stable, and would have a lifetime $\tau_{p} \simeq 10^{30}$ yrs. This went against the long-held idea of baryon number conservation, which had been proposed by Weyl to explain why neutrons decayed but protons did not~\citep{Weyl1929}. Proton decay had been considered since then, but there had never been any prediction of it. One of the earliest limits on the lifetime of the proton had actually been made by Maurice Goldhaber, who noted that if the proton lifetime was less than 10$^{18}$ we would receive lethal doses of radiation from its decay. \\

With the prediction of proton decay, experiments began searching for it in underground labs. The proton lifetime predicted by Georgi and Glashow has now been conclusively ruled out, and inconsistencies have been found with their original theory, such as the gauge couplings not unifying at a common energy, and the neutrinos predicted being massless. However, it began the search for a GUT, as well as the search for nucleon decay, and so it is interesting from an historical standpoint. Extensions to the original theory, which attempt to address some of the issues mentioned above, have been made, these extentsions predict a proton lifetime of $\tau_{p} \leq 10^{35}$ yrs. \\

The inconsistencies seen in the SU(5) model are rectified by the supersymmetric SU(5) model, which predicts super partners to the particles in the SM. In fact, it predicted the value of the Weinberg angle, and that the top quark would be $\simeq$200 GeV to an impressive accuracy. The initial proton lifetimes predicted were $\tau_{p} \simeq 10^{30-31}$ yrs, these lifetimes have now been ruled out. However, it is possible to get proton lifetimes of $\tau_{p} \simeq 10^{33-36}$ yrs, which are well within the current experimental limits. When attempting to explain neutrino mass and mixing, R-parity (as calculated in Equation~\ref{eq:RParity}) must be broken.
\begin{equation}
  \label{eq:RParity}
  P_{R} = (-1)^{3(B-L)+2s}
\end{equation}
where $B$ is baryon number, $L$ is lepton number, and $s$ is spin. SM particles have R parity +1, whilst supersymmetric particles have R-parity -1. The baryon - lepton number ($B-L$) is generally conserved in GUTs. However, after allowing for R-parity to be broken, additional channels of nucleon decay become possible, such as $n \rightarrow K^{+} + l^{-}$ and $p \rightarrow K^{+} + l^{-} + \pi^{+}$. It can be shown~\citep{Senjanovic:2009kr, Vissani:1995hp} that the decay rate of the $n \rightarrow K^{+} + l^{-}$ channel may be an order of magnitude larger than that of the proton. As will be shown in Table~\ref{tab:NDKLim}, these channels currently have very low limits on their lifetimes and so warrant further study, as nucleon decay signatures may potentially be observed with relatively low lifetimes. \\

Finally, it also possible to construct GUTs which use higher order symmetries, such as SO(10). When considering SO(10) theories, supersymmetric models are normally considered, as ordinary SO(10) models have failed to be able to predict fermion masses and mixings. However, the supersymmetric models are able to accommodate right hand neutrinos, explain the disparity between the quark and lepton mixing angles at SM energies, and predict the branching ratios of proton decays. \\

Over the years there have been many experiments which have searched for the signatures of nucleon decay, these have included NUSEX~\citep{BATTISTONI1983454}, FREJUS~\citep{berger:in2p3-00015565}, Soudan I/II~\citep{SoudanLim}, and IMB~\citep{Gajewski:1989gh}. The current experimental limit on the proton lifetime is held by the SK experiment, which sets a lower partial lifetime in the $p\ensuremath{\rightarrow}{e}^{+}{\ensuremath{\pi}}^{0}$ decay mode of 1.6 $\times$ 10$^{34}$ yrs~\citep{PhysRevD.95.012004}. A successor to SK, called Hyper-Kamiokande (HK), is planned for the mid 2020s~\citep{Abe:2011ts} and will increase the proton decay lifetime limits measured by SK by around an order of magnitude. DUNE will also search for nucleon decays, and will have a sensitivity which competes with HK despite being much smaller, due to the increased resolution of LArTPCs as compared to water Cherenkov detectors. The sensitivity of DUNE to nucleon decay will be discussed in Section~\ref{sec:DUNE_NDK}. \\  

%********************************** % Third Section  *************************************
\subsection{Backgrounds to nucleon decay} \label{sec:BkNDK}  %Section - X.2.3
In order to observe such rare processes it is necessary to place experiments in environments which have as little background as possible. It is for this reason that experiments which search for proton decay are placed far below the Earth's surface. This is to reduce the cosmic ray flux, which would render such searches impossible if they were attempted on the Earth's surface. At large depths the hadronic flux is totally suppressed, however cosmic ray muons can survive to depths of over a mile underground. These high energy cosmic muons are able to produce signal mimicking events in underground detectors. Therefore, in order to attempt to measure nucleon decays, these events have to be identified and rejected. \\

In the discussion of backgrounds to nucleon decay, reference will be given to the $p \rightarrow K^{+} \overline{\nu_{e}}$ decay channel, where the signal is a lone $K^{+}$ in the detector. The kaon is isolated as the neutrino is unlikely to interact before escaping the detector, and the recoiling nucleus would have too little energy to be detected. Though it is difficult to imagine a situation where a cosmic muon would directly produce an isolated kaon in the detector, one of its interaction products could do so. This is normally presented by imagining that a muon produces a $K^{0}_{L}$ outside of the detector, which then propagates into the detector, and undergoes charge exchange with a nucleus far away from the detector walls. This would produce an isolated kaon far away from the detector walls, with no associated tracks (the $K^{0}_{L}$ is neutral and so doesn't produce a track in LArTPCs). This event structure would look very similar to a nucleon decay event, and a schematic of such an interaction is shown in Figure~\ref{fig:K0LongBackground}. It is important to note that as LArTPCs are not magnetised it is very difficult to discern the charge of a kaon, and so a $K^{\pm}$ are largely indistinguishable. \\

\begin{figure}
  \centering
  \includegraphics[width=0.5\textwidth]{KaonNDKInteraction}
  \caption[Schematic of a nucleon decay mimicking events produced by a cosmic muon]
          {Schematic of a nucleon decay mimicking events produced by a cosmic muon. A cosmic muon produces a $K^{0}_{L}$ outside of the detector volume, which then interacts far from the detector wall producing an isolated kaon. There would be no other charged tracks produced, and so the event would be indistinguishable from a real proton decay event.}
  \label{fig:K0LongBackground}
\end{figure}

Atmospheric neutrinos are also able to produce nucleon decay mimicking signals if they interact deep within the active volume. However, the simulations contained in Chapter~\ref{chap:FDSims} concern backgrounds induced by cosmic rays, and so the backgrounds induced by atmospheric neutrinos will not be considered in this thesis. \\ 

\subsection{Kinematics of nucleon decay} \label{sec:NDKKin}
Though background inducing events like the one shown in Figure~\ref{fig:K0LongBackground} are rare, the nucleon decay rate is rarer still, and so it is necessary to find ways in which to identify background events as such. For this reason a fiducial cut is often applied, as many of the $K^{0}_{L}$s will interact close to the detector edge. Should an interaction pass this cut, it is possible to apply strict energy criteria to the reconstructed energies. This is because the energies of the particles produced in a nucleon decay event are well defined. \\

Through simple energy considerations, it is observed that the kaon produced by the $p \rightarrow K^{+} \overline{\nu_{e}}$ decay should have a momentum of about 340 MeV, and a total energy of about 600 MeV. The kaon would also be expected to decay at rest, and so the decay products from the kaon should have an energy equal to the rest mass of the kaon. \\

In Section~\ref{sec:DUNENDK} the background to the $n \rightarrow K^{+} e^{-}$ decay channel will be discussed. The kinematics of this decay are presented below, where the neutron is assumed to decay at rest.
\begin{align}
  E_{n} &= m_{n} = E_{K} + E_{e} \nonumber \\
  E_{e}^{2} &= (m_{n} - E_{K})^{2} \nonumber \\
  m_{e}^{2} + p_{e}^{2} &= m_{n}^{2} + (m_{K}^{2} + p_{K}^{2}) - 2m_{n}E_{K} \nonumber \\
  &\text{using conservation of momentum $\vec{p}_{K} = -\vec{p}_{e}$} \nonumber \\  
  E_{K} &= \frac{m_n^{2} + m_{K}^{2} - m_{e}^{2}}{2m_{n}} \label{eq:DecKEn} \\
  E_{e} &= \frac{m_n^{2} + m_{e}^{2} - m_{K}^{2}}{2m_{n}} \label{eq:DecEEn} \\
  \sqrt{E_{K}^{2} - m_{K}^{2}} = \vec{p_{K}} &= \sqrt{ \left(\frac{m_n^{2} + m_{e}^{2} - m_{K}^{2}}{2m_{n}}\right)^{2} - m_{K}^{2}\left(\frac{4m_{n}^2}{4m_n^{2}}\right) } \nonumber \\
  \vec{p_{K}} &= \frac{ \sqrt{m_n^{2} + m_{K}^{2} - m_{e}^{2} - 4m_{n}^{2}m_{K}^{2} } }{ 2m_{n} } \label{eq:DecEMom} \\
  \vec{p_{e}} &= \frac{ \sqrt{m_n^{2} + m_{e}^{2} - m_{K}^{2} - 4m_{n}^{2}m_{e}^{2} } }{ 2m_{n} } \label{eq:DecKMom}
\end{align}
Using $m_{n}$ = 939.565 MeV, $m_{K}$ = 493.677 MeV and $m_{e}$ = 0.511 MeV~\citep{PDGReview}, Equations~\ref{eq:DecEEn} and~\ref{eq:DecKEn} are equal to 599.479 MeV and 340.086 MeV respectively, whilst Equations~\ref{eq:DecEMom} and~\ref{eq:DecKMom} are both equal to 340.086 MeV. A kaon (electron) with $E_{K}$ = 599.479 ($E_{e}$ = 340.086) MeV will have a kinetic energy of 105.802 (339.575) MeV. \\

It is important to note that any nucleons which decay in LArTPCs will be contained in argon nuclei, this means that the energies of any particles which are produced will be smeared by the Fermi motion of the decaying nucleon within the nucleus. Any kaon which is produced in the decay is also likely to scatter as it exits the nucleus, further smearing its energy and momenta~\citep{Stefan:2008zi}. This causes the true momenta and total energy of the particles produced in the decay to be different from the values which were calculated above. The result of this is that when searching for nucleon decay events, it is necessary to consider energy ranges of a few hundred MeV around the values calculated above. \\
