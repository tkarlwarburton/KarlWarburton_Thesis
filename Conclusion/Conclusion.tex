%*******************************************************************************
%******************************** Chapter XXXXXXXX *****************************
%*******************************************************************************
\chapter{Concluding remarks} \label{chap:Conc} %Title of chapter

\graphicspath{{35tonData/Figs/PDF/}{35tonData/Figs/Raster/}{35tonData/Figs/Vector/}}

This thesis has included work done in connection to the DUNE single phase LArTPC design, with a strong emphasis on the Phase II run of 35~ton detector. Simulations have also been performed in relation to the cosmic background to neutrino interactions when considering a surface detector, and the muon-induced background to nucleon decay searches in an underground detector. \\

Chapter~\ref{chap:Cameras} concerned the installation and operation of a system of cameras in the 35~ton detector. These cameras were found to be fully operational, and proved to be an extremely useful monitoring tool during the Phase II run. However, they were unable to conclusively observe any high voltage breakdowns in the LAr, though this capability had been shown in tabletop tests. Nevertheless, there are plans to install similar systems in future LArTPCs such as SBND~\citep{SBNProposal}. \\

Chapter~\ref{chap:35tonSim} outlined the simulation efforts which were performed in preparation for the 35~ton Phase II run, culminating in a test of the method used to perform Particle IDentification (PID). The method of PID was seen to effectively identify muons and protons when they were simulated in the 35~ton detector in isolation. However, the method was seen to become less effective when the simulation was extended to include realistic samples of cosmic-ray events, where each ``event'' contained many particles. This was primarily because of the large number of reconstructed $\delta$ rays, which were associated with high energy muons passing through the detector. After the removal of these high energy muons, and the short tracks which were associated with them, a separation of proton and muon tracks was seen. This separation could be improved by using the true stopping point of the simulated particles from Monte Carlo truth. This showed that there was promise in performing the analysis on the data collected by the 35~ton data, though the selection of stopping particles would have to be fully developed in order for this to be possible, as otherwise through going muons would saturate any sample of stopping particles. \\

A discussion of the features and issues surrounding the 35~ton dataset were presented in Chapter~\ref{chap:35tonData}, concluding in the proposal of a new method of interaction time determination using the effects of longitudinal diffusion. Two metrics were proposed whereby this could be done, though considering the $RMS/Charge$ of hits in reconstructed tracks was found to be the better identifier, where the $RMS$ is the width of the reconstructed hit in ticks (500~ns), and the $Charge$ is the integrated area of the reconstructed hit (ADC). It was found that the interaction time of through-going tracks could be determined to an accuracy of 171~$\mu$s, where the distribution has a FWHM of 210~$\mu$s, over a drift window of around 5,200~$\mu$s. When this is converted to a prediction of the $x$ (drift) position of the particle, the accuracy is found to be 18.5~cm, with a FWHM of 23.0~cm, over a drift length of 223~cm. This was then compared to a simulated low-noise detector, where the accuracy of the interaction time ($x$ position) determination was found to be 3~$\mu$s (0.4~cm) with a FWHM of 114~$\mu$s (12.6~cm). The effect of changing detector conditions in Monte Carlo was then observed with relation to; the electronics noise level of the detector, the applied electric field, the electron lifetime, and the magnitude of the longitudinal diffusion constant. As expected, lower noise levels, and larger longitudinal diffusion constants were seen to increase the accuracy of the method. Lower electron lifetimes were seen to cause the method to be more effective, as this corresponds to larger decreases in hit charge, giving the method a stronger handle on the $x$ (drift) position of the particle. The method was seen to be largely unaffected by the electric field which was applied to the detector, this is reassuring as future detectors will run at 500~V$\cdot$cm$^{-1}$, and not at 250~V$\cdot$cm$^{-1}$ as the 35~ton detector did. \\

Finally, Chapter~\ref{chap:FDSims} concerns simulations of background events in LArTPCs. The backgrounds to neutrino interactions in a surface detector built on work presented in~\citep{MartinsThesis}, where a system of cuts designed to reject background events had been developed. The author's contribution concerned the addition of an accurate detector geometry, and an accurate surface profile to simulations. The accurate detector geometry was seen to significantly affect the predicted background rate, causing the hadronic flux to be reduced by an order of magnitude due to the addition of extra shielding, whilst the muon-induced background was seen to increase due to the detector having a larger surface area. The background rate was seen to be largely unaffected by the addition of the accurate surface profile. The combined annual background rate from muons, protons and neutrons, after all cuts were applied, was seen to be reduced from $\approx$5~events when using a simplified detector geometry and surface profile, to 3.07~$\pm$0.25~events when using the accurate detector geometry and surface profile. \\

Chapter~\ref{chap:FDSims} concluded with a study concerning the muon-induced background rate to the $n \rightarrow K^{+} + e^{-}$ decay channel. A series of cuts was developed to reject these events, concerning event features such as the proximity of hits to the detector walls, and the proximity of the kaon and electron tracks, as well as energy constraints assuming that the neutron decays at rest. After the cuts are applied to a sample of 2~$\times$~10$^{9}$ muons, representing 401.6~years of detector live time for a single DUNE far detector module, no events are found which would mimic a nucleon decay event. This corresponds to a limit on the background rate of less than 0.44 events$~\cdot$Mt$^{-1}\cdot$year$^{-1}$ at the 90\% confidence level, using double sided errors~\citep{PDGReview} and a fiducial mass of 13.8 kt to give an exposure of 5.542 Mt$\cdot$year. When the cuts are applied to a sample of 10,000 simulated signal events, a signal efficiency of 89.02\% is found, this rises to 96.8\% when only fully contained events are considered. This study suggests that DUNE is likely to be able to do a background free study of the $n \rightarrow K^{+} + e^{-}$ decay channel. However, the analysis did not consider backgrounds due to atmospheric neutrinos, and did not use any reconstructed quantities, as it only used Monte Carlo truth energy depositions from simulated cosmic muons, and so needs to be extended in the future. \\
