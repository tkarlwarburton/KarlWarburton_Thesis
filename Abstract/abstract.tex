% ************************** Thesis Abstract *****************************
% Use `abstract' as an option in the document class to print only the titlepage and the abstract.
\begin{abstract}

  The Deep Underground Neutrino Experiment (DUNE) is a next-generation neutrino experiment which will be built at the Sanford Underground Research Facility (SURF), and will receive a wide-band neutrino beam from Fermilab, 1300~km away. At this baseline DUNE will be able to study many of the properties of neutrino mixing, including the neutrino mass hierarchy and the value of the CP-violating complex phase ($\delta_{CP}$). DUNE will utilise Liquid Argon (LAr) Time Projection Chamber (TPC) (LArTPC) technology, and the Far Detector (FD) will consist of four modules, each containing 17.1~kt of LAr with a fiducial mass of around 10~kt. Each of these FD modules represents around an order of magnitude increase in size, when compared to existing LArTPC experiments. \\

  The 35 ton detector is the first DUNE prototype for the single (LAr) phase design of the FD. There were two running periods, one from November 2013 to February 2014, and a second from November 2015 to March 2016. During the second running period, a system of TPCs were installed, and cosmic-ray data were collected. A method of particle identification was developed using simulations, though this was not applied to the data due to the higher than expected noise level. A new method of determining the interaction time of a track, using the effects of longitudinal diffusion, was developed using the cosmic-ray data. A camera system was also installed in the detector for monitoring purposes, and to look for high voltage breakdowns. \\

  Simulations concerning the muon-induced background rate to nucleon decay are performed, following the incorporation of the MUon Simulations UNderground (MUSUN) generator into the DUNE software framework. A series of cuts which are based on Monte Carlo truth information is developed, designed to reject simulated background events, whilst preserving simulated signal events in the $n \rightarrow K^{+} + e^{-}$ decay channel. No background events are seen to survive the application of these cuts in a sample of 2~$\times$~10$^9$ muons, representing 401.6~years of detector live time. This corresponds to an annual background rate of <~0.44~events$\cdot$Mt$^{-1}\cdot$year$^{-1}$ at 90\% confidence, using a fiducial mass of 13.8~kt. \\
  
\end{abstract}
